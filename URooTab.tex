% Options for packages loaded elsewhere
\PassOptionsToPackage{unicode}{hyperref}
\PassOptionsToPackage{hyphens}{url}
\PassOptionsToPackage{dvipsnames,svgnames,x11names}{xcolor}
%
\documentclass[
  letterpaper,
  DIV=11,
  numbers=noendperiod]{scrartcl}

\usepackage{amsmath,amssymb}
\usepackage{iftex}
\ifPDFTeX
  \usepackage[T1]{fontenc}
  \usepackage[utf8]{inputenc}
  \usepackage{textcomp} % provide euro and other symbols
\else % if luatex or xetex
  \usepackage{unicode-math}
  \defaultfontfeatures{Scale=MatchLowercase}
  \defaultfontfeatures[\rmfamily]{Ligatures=TeX,Scale=1}
\fi
\usepackage{lmodern}
\ifPDFTeX\else  
    % xetex/luatex font selection
\fi
% Use upquote if available, for straight quotes in verbatim environments
\IfFileExists{upquote.sty}{\usepackage{upquote}}{}
\IfFileExists{microtype.sty}{% use microtype if available
  \usepackage[]{microtype}
  \UseMicrotypeSet[protrusion]{basicmath} % disable protrusion for tt fonts
}{}
\makeatletter
\@ifundefined{KOMAClassName}{% if non-KOMA class
  \IfFileExists{parskip.sty}{%
    \usepackage{parskip}
  }{% else
    \setlength{\parindent}{0pt}
    \setlength{\parskip}{6pt plus 2pt minus 1pt}}
}{% if KOMA class
  \KOMAoptions{parskip=half}}
\makeatother
\usepackage{xcolor}
\setlength{\emergencystretch}{3em} % prevent overfull lines
\setcounter{secnumdepth}{-\maxdimen} % remove section numbering
% Make \paragraph and \subparagraph free-standing
\ifx\paragraph\undefined\else
  \let\oldparagraph\paragraph
  \renewcommand{\paragraph}[1]{\oldparagraph{#1}\mbox{}}
\fi
\ifx\subparagraph\undefined\else
  \let\oldsubparagraph\subparagraph
  \renewcommand{\subparagraph}[1]{\oldsubparagraph{#1}\mbox{}}
\fi

\usepackage{color}
\usepackage{fancyvrb}
\newcommand{\VerbBar}{|}
\newcommand{\VERB}{\Verb[commandchars=\\\{\}]}
\DefineVerbatimEnvironment{Highlighting}{Verbatim}{commandchars=\\\{\}}
% Add ',fontsize=\small' for more characters per line
\usepackage{framed}
\definecolor{shadecolor}{RGB}{241,243,245}
\newenvironment{Shaded}{\begin{snugshade}}{\end{snugshade}}
\newcommand{\AlertTok}[1]{\textcolor[rgb]{0.68,0.00,0.00}{#1}}
\newcommand{\AnnotationTok}[1]{\textcolor[rgb]{0.37,0.37,0.37}{#1}}
\newcommand{\AttributeTok}[1]{\textcolor[rgb]{0.40,0.45,0.13}{#1}}
\newcommand{\BaseNTok}[1]{\textcolor[rgb]{0.68,0.00,0.00}{#1}}
\newcommand{\BuiltInTok}[1]{\textcolor[rgb]{0.00,0.23,0.31}{#1}}
\newcommand{\CharTok}[1]{\textcolor[rgb]{0.13,0.47,0.30}{#1}}
\newcommand{\CommentTok}[1]{\textcolor[rgb]{0.37,0.37,0.37}{#1}}
\newcommand{\CommentVarTok}[1]{\textcolor[rgb]{0.37,0.37,0.37}{\textit{#1}}}
\newcommand{\ConstantTok}[1]{\textcolor[rgb]{0.56,0.35,0.01}{#1}}
\newcommand{\ControlFlowTok}[1]{\textcolor[rgb]{0.00,0.23,0.31}{#1}}
\newcommand{\DataTypeTok}[1]{\textcolor[rgb]{0.68,0.00,0.00}{#1}}
\newcommand{\DecValTok}[1]{\textcolor[rgb]{0.68,0.00,0.00}{#1}}
\newcommand{\DocumentationTok}[1]{\textcolor[rgb]{0.37,0.37,0.37}{\textit{#1}}}
\newcommand{\ErrorTok}[1]{\textcolor[rgb]{0.68,0.00,0.00}{#1}}
\newcommand{\ExtensionTok}[1]{\textcolor[rgb]{0.00,0.23,0.31}{#1}}
\newcommand{\FloatTok}[1]{\textcolor[rgb]{0.68,0.00,0.00}{#1}}
\newcommand{\FunctionTok}[1]{\textcolor[rgb]{0.28,0.35,0.67}{#1}}
\newcommand{\ImportTok}[1]{\textcolor[rgb]{0.00,0.46,0.62}{#1}}
\newcommand{\InformationTok}[1]{\textcolor[rgb]{0.37,0.37,0.37}{#1}}
\newcommand{\KeywordTok}[1]{\textcolor[rgb]{0.00,0.23,0.31}{#1}}
\newcommand{\NormalTok}[1]{\textcolor[rgb]{0.00,0.23,0.31}{#1}}
\newcommand{\OperatorTok}[1]{\textcolor[rgb]{0.37,0.37,0.37}{#1}}
\newcommand{\OtherTok}[1]{\textcolor[rgb]{0.00,0.23,0.31}{#1}}
\newcommand{\PreprocessorTok}[1]{\textcolor[rgb]{0.68,0.00,0.00}{#1}}
\newcommand{\RegionMarkerTok}[1]{\textcolor[rgb]{0.00,0.23,0.31}{#1}}
\newcommand{\SpecialCharTok}[1]{\textcolor[rgb]{0.37,0.37,0.37}{#1}}
\newcommand{\SpecialStringTok}[1]{\textcolor[rgb]{0.13,0.47,0.30}{#1}}
\newcommand{\StringTok}[1]{\textcolor[rgb]{0.13,0.47,0.30}{#1}}
\newcommand{\VariableTok}[1]{\textcolor[rgb]{0.07,0.07,0.07}{#1}}
\newcommand{\VerbatimStringTok}[1]{\textcolor[rgb]{0.13,0.47,0.30}{#1}}
\newcommand{\WarningTok}[1]{\textcolor[rgb]{0.37,0.37,0.37}{\textit{#1}}}

\providecommand{\tightlist}{%
  \setlength{\itemsep}{0pt}\setlength{\parskip}{0pt}}\usepackage{longtable,booktabs,array}
\usepackage{calc} % for calculating minipage widths
% Correct order of tables after \paragraph or \subparagraph
\usepackage{etoolbox}
\makeatletter
\patchcmd\longtable{\par}{\if@noskipsec\mbox{}\fi\par}{}{}
\makeatother
% Allow footnotes in longtable head/foot
\IfFileExists{footnotehyper.sty}{\usepackage{footnotehyper}}{\usepackage{footnote}}
\makesavenoteenv{longtable}
\usepackage{graphicx}
\makeatletter
\def\maxwidth{\ifdim\Gin@nat@width>\linewidth\linewidth\else\Gin@nat@width\fi}
\def\maxheight{\ifdim\Gin@nat@height>\textheight\textheight\else\Gin@nat@height\fi}
\makeatother
% Scale images if necessary, so that they will not overflow the page
% margins by default, and it is still possible to overwrite the defaults
% using explicit options in \includegraphics[width, height, ...]{}
\setkeys{Gin}{width=\maxwidth,height=\maxheight,keepaspectratio}
% Set default figure placement to htbp
\makeatletter
\def\fps@figure{htbp}
\makeatother
\newlength{\cslhangindent}
\setlength{\cslhangindent}{1.5em}
\newlength{\csllabelwidth}
\setlength{\csllabelwidth}{3em}
\newlength{\cslentryspacingunit} % times entry-spacing
\setlength{\cslentryspacingunit}{\parskip}
\newenvironment{CSLReferences}[2] % #1 hanging-ident, #2 entry spacing
 {% don't indent paragraphs
  \setlength{\parindent}{0pt}
  % turn on hanging indent if param 1 is 1
  \ifodd #1
  \let\oldpar\par
  \def\par{\hangindent=\cslhangindent\oldpar}
  \fi
  % set entry spacing
  \setlength{\parskip}{#2\cslentryspacingunit}
 }%
 {}
\usepackage{calc}
\newcommand{\CSLBlock}[1]{#1\hfill\break}
\newcommand{\CSLLeftMargin}[1]{\parbox[t]{\csllabelwidth}{#1}}
\newcommand{\CSLRightInline}[1]{\parbox[t]{\linewidth - \csllabelwidth}{#1}\break}
\newcommand{\CSLIndent}[1]{\hspace{\cslhangindent}#1}

\KOMAoption{captions}{tableheading}
\makeatletter
\makeatother
\makeatletter
\makeatother
\makeatletter
\@ifpackageloaded{caption}{}{\usepackage{caption}}
\AtBeginDocument{%
\ifdefined\contentsname
  \renewcommand*\contentsname{Table of contents}
\else
  \newcommand\contentsname{Table of contents}
\fi
\ifdefined\listfigurename
  \renewcommand*\listfigurename{List of Figures}
\else
  \newcommand\listfigurename{List of Figures}
\fi
\ifdefined\listtablename
  \renewcommand*\listtablename{List of Tables}
\else
  \newcommand\listtablename{List of Tables}
\fi
\ifdefined\figurename
  \renewcommand*\figurename{Figure}
\else
  \newcommand\figurename{Figure}
\fi
\ifdefined\tablename
  \renewcommand*\tablename{Table}
\else
  \newcommand\tablename{Table}
\fi
}
\@ifpackageloaded{float}{}{\usepackage{float}}
\floatstyle{ruled}
\@ifundefined{c@chapter}{\newfloat{codelisting}{h}{lop}}{\newfloat{codelisting}{h}{lop}[chapter]}
\floatname{codelisting}{Listing}
\newcommand*\listoflistings{\listof{codelisting}{List of Listings}}
\makeatother
\makeatletter
\@ifpackageloaded{caption}{}{\usepackage{caption}}
\@ifpackageloaded{subcaption}{}{\usepackage{subcaption}}
\makeatother
\makeatletter
\@ifpackageloaded{tcolorbox}{}{\usepackage[skins,breakable]{tcolorbox}}
\makeatother
\makeatletter
\@ifundefined{shadecolor}{\definecolor{shadecolor}{rgb}{.97, .97, .97}}
\makeatother
\makeatletter
\makeatother
\makeatletter
\makeatother
\ifLuaTeX
  \usepackage{selnolig}  % disable illegal ligatures
\fi
\IfFileExists{bookmark.sty}{\usepackage{bookmark}}{\usepackage{hyperref}}
\IfFileExists{xurl.sty}{\usepackage{xurl}}{} % add URL line breaks if available
\urlstyle{same} % disable monospaced font for URLs
\hypersetup{
  pdftitle={URooTab: Tabular Reporting of EViews Unit Root Tests},
  pdfauthor={Sagiru Mati},
  colorlinks=true,
  linkcolor={blue},
  filecolor={Maroon},
  citecolor={Blue},
  urlcolor={Blue},
  pdfcreator={LaTeX via pandoc}}

\title{URooTab: Tabular Reporting of EViews Unit Root Tests}
\author{Sagiru Mati}
\date{2023-08-30}

\begin{document}
\maketitle
\ifdefined\Shaded\renewenvironment{Shaded}{\begin{tcolorbox}[boxrule=0pt, breakable, interior hidden, sharp corners, enhanced, frame hidden, borderline west={3pt}{0pt}{shadecolor}]}{\end{tcolorbox}}\fi

Please do not forget to cite the package as follows:

\textbf{Plain text:}

\textbf{Mati S. (2023). URooTab: Tabular Reporting of EViews Unit Root
Tests. CRAN, https://github.com/sagirumati/URooTab}

\textbf{Bibtex:}

\begin{verbatim}
  @Manual{Mati2023,
    title = {{URooTab}: Tabular Reporting of {EViews} Unit Root Tests},
    author = {Sagiru Mati},
    publisher = {CRAN},
    url = {https://github.com/sagirumati/URooTab},
  }
\end{verbatim}

\hypertarget{about-the-author}{%
\section{About the Author}\label{about-the-author}}

The author of this package, \textbf{Sagiru Mati}, obtained his PhD in
Economics from the Near East University, North Cyprus. He works at the
Department of Economics, Yusuf Maitama Sule (Northwest) University,
Kano, Nigeria. Please visit his \href{https://smati.com.ng}{website} for
more details.

Please follow his publications on:

\href{https://scholar.google.com/citations?user=odEp1eIAAAAJ\&hl=en\&oi=ao}{\textbf{Google
Scholar}}

\href{https://www.researchgate.net/profile/Sagiru-Mati-2}{\textbf{ResearchGate}}

\href{https://www.webofscience.com/wos/author/record/P-3408-2017}{\textbf{Web
of Science}}

\href{https://orcid.org/0000-0003-1413-3974}{\textbf{ORCID:
0000-0003-1413-3974}}

\hypertarget{about-urootab}{%
\section{About URooTab}\label{about-urootab}}

URooTab is an R package that can conducts \texttt{EViews} unit root
tests and report them in tabular form.

\hypertarget{why-urootab}{%
\section{Why URooTab?}\label{why-urootab}}

While there are R packages and EViews add-ins available for presenting
unit root tests in tabular form, none of them incorporates
\textbf{EViews} procedures within the R environment. Specifically:

\begin{itemize}
\item
  I wish I could conduct unit root using EViews routines in R, R
  Markdown or Quarto document
\item
  I wish I could dynamically import the results of the unit root tests
  individually or at once into R, R Markdown or Quarto document without
  switching between these applications back and forth.
\item
  I wish I could use an R function to report unit root test in a table
  style suitable for publication.
\item
  I wish I could automatically format the table in \texttt{Latex},
  \texttt{html}, \texttt{pandoc} and \texttt{markdown}.
\item
  I wish I could do all of the above from R, R Markdown or Quarto
  without opening the EViews!!!
\end{itemize}

\hypertarget{installation}{%
\section{Installation}\label{installation}}

URooTab can be installed using the following commands in R.

\begin{Shaded}
\begin{Highlighting}[]
\InformationTok{\textasciigrave{}\textasciigrave{}\textasciigrave{}\{r installation,eval=F\}}
\FunctionTok{install.packages}\NormalTok{(}\StringTok{"URooTab"}\NormalTok{) }

\NormalTok{            OR}
            
\NormalTok{devtools}\SpecialCharTok{::}\FunctionTok{install\_github}\NormalTok{(}\StringTok{\textquotesingle{}sagirumati/URooTab\textquotesingle{}}\NormalTok{)}
\InformationTok{\textasciigrave{}\textasciigrave{}\textasciigrave{}}
\end{Highlighting}
\end{Shaded}

\hypertarget{setup}{%
\section{Setup}\label{setup}}

To run the package successfully, you need to do one of the following

\begin{itemize}
\item
  Add EViews installation folder to path (\textbf{Environment
  Variables}).
\item
  Don't do anything if the name of EViews executable is one of the
  following: \texttt{EViews13\_x64}, \texttt{EViews13\_x86},
  \texttt{EViews12\_x64}, \texttt{EViews12\_x86},
  \texttt{EViews11\_x64}, \texttt{EViews11\_x86},
  \texttt{EViews10\_x64}, \texttt{EViews10\_x86}, \texttt{EViews9\_x64},
  \texttt{EViews9\_x86}, \texttt{EViews10}. The package will find the
  executable automatically.
\item
  Rename the Eviews executable to \texttt{eviews} or one of the names
  above.
\item
  Alternatively, you can use \texttt{set\_eviews\_path()} function to
  set the path the EViews executable as follows:
\end{itemize}

\begin{Shaded}
\begin{Highlighting}[]
\InformationTok{\textasciigrave{}\textasciigrave{}\textasciigrave{}\{r eval=F\}}
\FunctionTok{library}\NormalTok{(EviewR)}
\FunctionTok{set\_eviews\_path}\NormalTok{(}\StringTok{"C:/Program Files (x86)/EViews 10/EViews10.exe"}\NormalTok{)}
\InformationTok{\textasciigrave{}\textasciigrave{}\textasciigrave{}}
\end{Highlighting}
\end{Shaded}

\hypertarget{usage}{%
\section{Usage}\label{usage}}

Please load the URooTab package as follows:

\begin{verbatim}
```{r}                                                                .
library(URooTab)
```
\end{verbatim}

\hypertarget{ways-to-use-urootab}{%
\section{Ways to use URooTab}\label{ways-to-use-urootab}}

The package can work with base R, R Markdown or Quarto document.

\hypertarget{urootab-along-with-r-markdown-or-quarto-document}{%
\section{URooTab along with R Markdown or Quarto
document}\label{urootab-along-with-r-markdown-or-quarto-document}}

You can use \texttt{URooTab} in an R chunk in R Markdown or Quarto
document:

The \texttt{uroot()} function reports all the available test (ADF and
PP) at once. It is more suitable for Quarto document, which has both
\texttt{tbl-cap} and \texttt{tbl-subcap} chunk options.

To produce Table @ref(tab:URooTab), use the R chunk below:

Notice the chunk option \texttt{results:\ asis} because \texttt{uroot()}
is designed to print all the tables (ADF and PP) in the chunk. If you
are producing multiple \texttt{kable} tables, \texttt{results:\ asis} is
necessary. You can also use \texttt{kableExtra} package to further
customise the table.

\begin{verbatim}
```{r}
#| label: URooTab
#| eval: true
#| results: asis
    
library(URooTab)
set.seed(1234) # for reproducibility
x=rnorm(100)
y=cumsum(x)
z=cumsum(y)

dataFrame=data.frame(x,y,z)
uroot(dataFrame, caption = "Unit Root Tests for x, y and Z")
```
\end{verbatim}

\begin{longtable}[]{@{}
  >{\raggedright\arraybackslash}p{(\columnwidth - 14\tabcolsep) * \real{0.1031}}
  >{\raggedright\arraybackslash}p{(\columnwidth - 14\tabcolsep) * \real{0.1031}}
  >{\raggedright\arraybackslash}p{(\columnwidth - 14\tabcolsep) * \real{0.1031}}
  >{\raggedright\arraybackslash}p{(\columnwidth - 14\tabcolsep) * \real{0.1959}}
  >{\raggedright\arraybackslash}p{(\columnwidth - 14\tabcolsep) * \real{0.1031}}
  >{\raggedright\arraybackslash}p{(\columnwidth - 14\tabcolsep) * \real{0.1031}}
  >{\raggedright\arraybackslash}p{(\columnwidth - 14\tabcolsep) * \real{0.1959}}
  >{\raggedright\arraybackslash}p{(\columnwidth - 14\tabcolsep) * \real{0.0928}}@{}}
\caption{Unit Root Tests for x, y and Z}\tabularnewline
\toprule\noalign{}
\begin{minipage}[b]{\linewidth}\raggedright
Variables
\end{minipage} & \begin{minipage}[b]{\linewidth}\raggedright
None
\end{minipage} & \begin{minipage}[b]{\linewidth}\raggedright
Constant
\end{minipage} & \begin{minipage}[b]{\linewidth}\raggedright
Constant and trend
\end{minipage} & \begin{minipage}[b]{\linewidth}\raggedright
None
\end{minipage} & \begin{minipage}[b]{\linewidth}\raggedright
Constant
\end{minipage} & \begin{minipage}[b]{\linewidth}\raggedright
Constant and trend
\end{minipage} & \begin{minipage}[b]{\linewidth}\raggedright
Decision
\end{minipage} \\
\midrule\noalign{}
\endfirsthead
\toprule\noalign{}
\begin{minipage}[b]{\linewidth}\raggedright
Variables
\end{minipage} & \begin{minipage}[b]{\linewidth}\raggedright
None
\end{minipage} & \begin{minipage}[b]{\linewidth}\raggedright
Constant
\end{minipage} & \begin{minipage}[b]{\linewidth}\raggedright
Constant and trend
\end{minipage} & \begin{minipage}[b]{\linewidth}\raggedright
None
\end{minipage} & \begin{minipage}[b]{\linewidth}\raggedright
Constant
\end{minipage} & \begin{minipage}[b]{\linewidth}\raggedright
Constant and trend
\end{minipage} & \begin{minipage}[b]{\linewidth}\raggedright
Decision
\end{minipage} \\
\midrule\noalign{}
\endhead
\bottomrule\noalign{}
\endlastfoot
X & -8.300*** & -8.396*** & -8.815*** & -8.274*** & -8.239*** &
-8.214*** & I(0) \\
Y & 0.417 & -1.907 & 0.026 & -8.148*** & -8.259*** & -8.721*** & I(1) \\
Z & -2.379** & -2.084 & -2.938 & 0.417 & -2.013 & -0.033 & I(2) \\
\end{longtable}

\begin{longtable}[]{@{}
  >{\raggedright\arraybackslash}p{(\columnwidth - 14\tabcolsep) * \real{0.1010}}
  >{\raggedright\arraybackslash}p{(\columnwidth - 14\tabcolsep) * \real{0.1010}}
  >{\raggedright\arraybackslash}p{(\columnwidth - 14\tabcolsep) * \real{0.1010}}
  >{\raggedright\arraybackslash}p{(\columnwidth - 14\tabcolsep) * \real{0.1919}}
  >{\raggedright\arraybackslash}p{(\columnwidth - 14\tabcolsep) * \real{0.1111}}
  >{\raggedright\arraybackslash}p{(\columnwidth - 14\tabcolsep) * \real{0.1111}}
  >{\raggedright\arraybackslash}p{(\columnwidth - 14\tabcolsep) * \real{0.1919}}
  >{\raggedright\arraybackslash}p{(\columnwidth - 14\tabcolsep) * \real{0.0909}}@{}}
\caption{Unit Root Tests for x, y and Z}\tabularnewline
\toprule\noalign{}
\begin{minipage}[b]{\linewidth}\raggedright
Variables
\end{minipage} & \begin{minipage}[b]{\linewidth}\raggedright
None
\end{minipage} & \begin{minipage}[b]{\linewidth}\raggedright
Constant
\end{minipage} & \begin{minipage}[b]{\linewidth}\raggedright
Constant and trend
\end{minipage} & \begin{minipage}[b]{\linewidth}\raggedright
None
\end{minipage} & \begin{minipage}[b]{\linewidth}\raggedright
Constant
\end{minipage} & \begin{minipage}[b]{\linewidth}\raggedright
Constant and trend
\end{minipage} & \begin{minipage}[b]{\linewidth}\raggedright
Decision
\end{minipage} \\
\midrule\noalign{}
\endfirsthead
\toprule\noalign{}
\begin{minipage}[b]{\linewidth}\raggedright
Variables
\end{minipage} & \begin{minipage}[b]{\linewidth}\raggedright
None
\end{minipage} & \begin{minipage}[b]{\linewidth}\raggedright
Constant
\end{minipage} & \begin{minipage}[b]{\linewidth}\raggedright
Constant and trend
\end{minipage} & \begin{minipage}[b]{\linewidth}\raggedright
None
\end{minipage} & \begin{minipage}[b]{\linewidth}\raggedright
Constant
\end{minipage} & \begin{minipage}[b]{\linewidth}\raggedright
Constant and trend
\end{minipage} & \begin{minipage}[b]{\linewidth}\raggedright
Decision
\end{minipage} \\
\midrule\noalign{}
\endhead
\bottomrule\noalign{}
\endlastfoot
X & -8.327*** & -8.418*** & -8.815*** & -42.502*** & -51.961*** &
-74.206*** & I(0) \\
Y & 0.275 & -1.857 & -0.066 & -8.170*** & -8.275*** & -8.721*** &
I(1) \\
Z & 6.659 & 3.450 & -3.516** & 0.274 & -1.956 & -0.109 & I(2) \\
\end{longtable}

In R Markdown or Quarto document, \texttt{URooTab} is smart enough to
recognise the document format and select the suitable table format.

\hypertarget{urootab-along-with-base-r.}{%
\section{URooTab along with base R.}\label{urootab-along-with-base-r.}}

In base R, you can get the table printed in console in the format you
specify by the \texttt{format} argument.

We can create a dataframe as follows:

\begin{Shaded}
\begin{Highlighting}[]
\FunctionTok{library}\NormalTok{(URooTab)}
\FunctionTok{set.seed}\NormalTok{(}\DecValTok{1234}\NormalTok{) }\CommentTok{\# for reproducibility}
\NormalTok{x}\OtherTok{=}\FunctionTok{rnorm}\NormalTok{(}\DecValTok{100}\NormalTok{) }
\NormalTok{y}\OtherTok{=}\FunctionTok{cumsum}\NormalTok{(x)}
\NormalTok{z}\OtherTok{=}\FunctionTok{cumsum}\NormalTok{(y)}

\NormalTok{dataFrame}\OtherTok{=}\FunctionTok{data.frame}\NormalTok{(x,y,z)}
\end{Highlighting}
\end{Shaded}

\hypertarget{the-adf-function}{%
\subsection{The adf() function}\label{the-adf-function}}

To print ADF test results in \texttt{latex} format:

\begin{Shaded}
\begin{Highlighting}[]
\InformationTok{\textasciigrave{}\textasciigrave{}\textasciigrave{}\{r\}}
\CommentTok{\#| label: adf}
\CommentTok{\#| eval: false}

\FunctionTok{adf}\NormalTok{(dataFrame,}\AttributeTok{format =} \StringTok{"latex"}\NormalTok{,}\AttributeTok{info=}\StringTok{"aic"}\NormalTok{,}
    \AttributeTok{caption =} \StringTok{"ADF Unit Root Tests for x, y and Z"}\NormalTok{) }
\InformationTok{\textasciigrave{}\textasciigrave{}\textasciigrave{}}
\end{Highlighting}
\end{Shaded}

Or

\begin{Shaded}
\begin{Highlighting}[]
\InformationTok{\textasciigrave{}\textasciigrave{}\textasciigrave{}\{r\}}
\CommentTok{\#| label: adf1}
\CommentTok{\#| eval: false}

\FunctionTok{uroot}\NormalTok{(dataFrame,}\AttributeTok{format =} \StringTok{"latex"}\NormalTok{,}\AttributeTok{test =} \StringTok{"adf"}\NormalTok{,}\AttributeTok{info=}\StringTok{"aic"}\NormalTok{,}
      \AttributeTok{caption =} \StringTok{"ADF Unit Root Tests for x, y and Z"}\NormalTok{)}
\InformationTok{\textasciigrave{}\textasciigrave{}\textasciigrave{}}
\end{Highlighting}
\end{Shaded}

The above code produces the following latex code:

\begin{verbatim}
\begin{table}[h]

\caption{ADF Unit Root Tests for x, y and Z}
\centering
\begin{tabular}[t]{llllllll}
\toprule
Variables & None & Constant & Constant and trend & None & Constant & Constant and trend & Decision\\
\midrule
X & -8.300*** & -8.396*** & -8.815*** & -7.494*** & -7.460*** & -7.445*** & I(0)\\
Y & 0.224 & -1.934 & 0.026 & -8.148*** & -8.259*** & -8.721*** & I(1)\\
Z & -2.379** & -2.084 & -2.938 & 0.233 & -2.221 & -0.033 & I(2)\\
\bottomrule
\end{tabular}
\end{table}
\end{verbatim}

\hypertarget{the-pp-function}{%
\subsection{The pp() function}\label{the-pp-function}}

To print PP test results in \texttt{html} format:

\begin{Shaded}
\begin{Highlighting}[]
\InformationTok{\textasciigrave{}\textasciigrave{}\textasciigrave{}\{r\}}
\CommentTok{\#| label: pp}
\CommentTok{\#| eval: false}

\FunctionTok{pp}\NormalTok{(dataFrame,}\AttributeTok{format =} \StringTok{"html"}\NormalTok{,}\AttributeTok{info=}\StringTok{"aic"}\NormalTok{,}\AttributeTok{caption =} \StringTok{"PP Unit Root Tests for x, y and Z"}\NormalTok{)}
\InformationTok{\textasciigrave{}\textasciigrave{}\textasciigrave{}}
\end{Highlighting}
\end{Shaded}

Or

\begin{Shaded}
\begin{Highlighting}[]
\InformationTok{\textasciigrave{}\textasciigrave{}\textasciigrave{}\{r\}}
\CommentTok{\#| label: pp1}
\CommentTok{\#| eval: false}

\FunctionTok{uroot}\NormalTok{(dataFrame,}\AttributeTok{format =} \StringTok{"html"}\NormalTok{,}\AttributeTok{info=}\StringTok{"aic"}\NormalTok{,}\AttributeTok{test =} \StringTok{"pp"}\NormalTok{,}
      \AttributeTok{caption =} \StringTok{"PP Unit Root Tests for x, y and Z"}\NormalTok{)}
\InformationTok{\textasciigrave{}\textasciigrave{}\textasciigrave{}}
\end{Highlighting}
\end{Shaded}

The above code produces the following \texttt{html} codes in console:

\begin{verbatim}
<table>
<caption>PP Unit Root Tests for x, y and Z</caption>
 <thead>
  <tr>
   <th style="text-align:left;"> Variables </th>
   <th style="text-align:left;"> None </th>
   <th style="text-align:left;"> Constant </th>
   <th style="text-align:left;"> Constant and trend </th>
   <th style="text-align:left;"> None </th>
   <th style="text-align:left;"> Constant </th>
   <th style="text-align:left;"> Constant and trend </th>
   <th style="text-align:left;"> Decision </th>
  </tr>
 </thead>
<tbody>
  <tr>
   <td style="text-align:left;"> X </td>
   <td style="text-align:left;"> -8.327*** </td>
   <td style="text-align:left;"> -8.418*** </td>
   <td style="text-align:left;"> -8.815*** </td>
   <td style="text-align:left;"> -42.502*** </td>
   <td style="text-align:left;"> -51.961*** </td>
   <td style="text-align:left;"> -74.206*** </td>
   <td style="text-align:left;"> I(0) </td>
  </tr>
  <tr>
   <td style="text-align:left;"> Y </td>
   <td style="text-align:left;"> 0.275 </td>
   <td style="text-align:left;"> -1.857 </td>
   <td style="text-align:left;"> -0.066 </td>
   <td style="text-align:left;"> -8.170*** </td>
   <td style="text-align:left;"> -8.275*** </td>
   <td style="text-align:left;"> -8.721*** </td>
   <td style="text-align:left;"> I(1) </td>
  </tr>
  <tr>
   <td style="text-align:left;"> Z </td>
   <td style="text-align:left;"> 6.659 </td>
   <td style="text-align:left;"> 3.450 </td>
   <td style="text-align:left;"> -3.516** </td>
   <td style="text-align:left;"> 0.274 </td>
   <td style="text-align:left;"> -1.956 </td>
   <td style="text-align:left;"> -0.109 </td>
   <td style="text-align:left;"> I(2) </td>
  </tr>
</tbody>
</table>
\end{verbatim}

\hypertarget{the-uroot-function}{%
\subsection{The uroot() function}\label{the-uroot-function}}

The \texttt{uroot()} function is a generic function that can be used to
conduct any unit root test. Setting \texttt{test="adf"} conducts ADF
test, while \texttt{test="pp"} conducts PP test. If \texttt{test}
argument is not specified, the \texttt{uroot()} function conducts all
the test at once.

\hypertarget{similar-packages}{%
\section{Similar Packages}\label{similar-packages}}

Similar packages include
\href{https://github.com/sagirumati/DynareR}{DynareR} (Mati 2020a,
2022a), \href{https://github.com/sagirumati/gretlR}{gretlR} (Mati 2020c,
2022c), and \href{https://github.com/sagirumati/EviewsR}{EviewsR} (Mati
2022b, 2020b; Mati, Civcir, and Abba 2023)

For further details, consult Mati (2023b) and Mati (2023a).

\hypertarget{references}{%
\section*{References}\label{references}}
\addcontentsline{toc}{section}{References}

\hypertarget{refs}{}
\begin{CSLReferences}{1}{0}
\leavevmode\vadjust pre{\hypertarget{ref-Mati2020}{}}%
Mati, Sagiru. 2020a. {``DynareR: Bringing the Power of Dynare to {R, R
Markdown, and Quarto}.''} \emph{CRAN}.
\url{https://CRAN.R-project.org/package=DynareR}.

\leavevmode\vadjust pre{\hypertarget{ref-Mati2020a}{}}%
---------. 2020b. \emph{EviewsR: A Seamless Integration of {EViews} and
{R}}. \url{https://CRAN.R-project.org/package=EviewsR}.

\leavevmode\vadjust pre{\hypertarget{ref-Mati2020b}{}}%
---------. 2020c. \emph{gretlR: A Seamless Integration of {Gretl} and
{R}}. \url{https://CRAN.R-project.org/package=gretlR}.

\leavevmode\vadjust pre{\hypertarget{ref-mati2022dynarer}{}}%
---------. 2022a. {``Package {`DynareR'}.''}
\url{https://cran.r-project.org/web/packages/DynareR/DynareR.pdf}.

\leavevmode\vadjust pre{\hypertarget{ref-mati2022eviewsr}{}}%
---------. 2022b. {``Package {`EviewsR'}.''}
\url{https://cran.r-project.org/web/packages/EviewsR/EviewsR.pdf}.

\leavevmode\vadjust pre{\hypertarget{ref-mati2022gretlr}{}}%
---------. 2022c. {``Package {`gretlR'}.''}
\url{https://cran.r-project.org/web/packages/gretlR/gretlR.pdf}.

\leavevmode\vadjust pre{\hypertarget{ref-mati2023urootab}{}}%
---------. 2023a. {``Package {`URooTab'}.''}
\url{https://cran.r-project.org/web/packages/URooTab/URooTab.pdf}.

\leavevmode\vadjust pre{\hypertarget{ref-Mati2023a}{}}%
---------. 2023b. \emph{{URooTab}: Tabular Reporting of {EViews} Unit
Root Tests}. \url{https://github.com/sagirumati/URooTab}.

\leavevmode\vadjust pre{\hypertarget{ref-Mati2023}{}}%
Mati, Sagiru, Irfan Civcir, and S. I. Abba. 2023. {``{EviewsR}: An r
Package for Dynamic and Reproducible Research Using {EViews}, r, r
Markdown and Quarto.''} \emph{The R Journal} 15 (2): 169--205.
\url{https://doi.org/10.32614/rj-2023-045}.

\end{CSLReferences}



\end{document}
